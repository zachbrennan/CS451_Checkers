%---------------------------------------------------------------
% File information.

% Filename : docs/release.tex
% Purpose  : Multiplayer, networked Checkers game for CS451.
% Authors  : Corwin Belser <cmb539@drexel.edu>
%            Zach Brennan  < zab37@drexel.edu>
%            Kris Horsey   < kth37@drexel.edu>
%            Zach van Rijn < zwv23@drexel.edu>
% License  : MIT/X (excl. ext. libs; see respective licenses).
% Revision : 20170829

%---------------------------------------------------------------
% README.

% This is currently a DRAFT.
%
% Also please keep it to NO MORE THAN 64 columns. This isn't a
% requirement by LaTeX, but a guideline we are following for the
% entire project. Thanks.
%
% You will need some kind of LaTeX compiler. Recommend this:
% https://www.tug.org/texlive/acquire-netinstall.html
% which works on Windows and Linux.
%
% The `vhistory' package has nice documentation:
% http://mirrors.rit.edu\
% /CTAN/macros/latex/contrib/vhistory/doc/vh_sets_en.pdf
%
% And this page on making tables:
% https://en.wikibooks.org/wiki/LaTeX/Tables

%---------------------------------------------------------------
% Includes.

\documentclass[letterpaper]{article}

\usepackage{amssymb}                    % for keyword styling
\usepackage[english]{babel}
\usepackage{bookmark}                   % tracks .out files
\usepackage{cleveref}                   % \cref
\usepackage[usenames,dvipsnames]{color} % for colors
\usepackage{courier}                    % \ttfamily
\usepackage{listings}                   % \lstlisting, \lstset
\usepackage[useregional]{datetime2}
\usepackage[numbers, square]{natbib}    % for proper \bibsection
\usepackage{rotating}                   % for sidewaysfigure
\usepackage{subcaption}                 % for nested figures
\usepackage{tabu}                       % flexible tables
\usepackage{tabulary}                   % \tabulary (more flex.)

%---------------------------------------------------------------
% Configuration.

\renewcommand{\thefootnote}{\arabic{footnote}}

\title{
    Console-Based Checkers Game\\
    Release Notes
    \footnote{CS451:002 Group 2, Drexel University}
}

% \usepackage{authblk} causes problems with alignment, though it
% also allows us to properly include affiliations.. pick poison.
\author{
    Belser, C.\\
    \texttt{cmb923@drexel.edu}
    \and
    Brennan, Z.\\
    \texttt{zab37@drexel.edu}
    \and
    Horsey, K.\\
    \texttt{kth37@drexel.edu}
    \and
    van Rijn, Z.\\
    \texttt{zwv23@drexel.edu}
}

\date{\today}

%---------------------------------------------------------------
% Document begin.

\begin{document}

%---------------------------------------------------------------
% Header.

\maketitle

\begin{abstract}

Thank you for your purchase of and patience with your new
Console-Based Checkers(tm) game. We hope you will be thoroughly
satisfied with our implementation, and look forward to serving
you and your business needs in the future. Please don't hesitate
to reach out to us in the unlikely event that you encounter any
issues with your new software, or wish for us to implement
additional features.

\end{abstract}

%---------------------------------------------------------------
% Source Code.

\section{Source Code}

You will find the complete source code including helpful scripts
and utilities in the included \texttt{final\_transfer.tar.gz}.
We employ the following directory structure:

\lstset{basicstyle=\ttfamily}
\lstset{frame=tb}
\begin{lstlisting}
Z:\PROJECTS\CS451-CHECKERS
+---docker
|   +---Dockerfile
|   \---Makefile
+---docs
|   +---design.pdf
|   +---design.tex
|   +---img
|   |   +---crown120.png
|   |   +---crown.png
|   |   +---logo.png
|   |   +---old
|   |   |   +---GUI_Loading_Screen
|   |   |   +---UML_Client
|   |   |   +---UML_Client.png
|   |   |   +---UML_Game_Logic
|   |   |   +---UML_Game_Logic.png
|   |   |   +---UML_Game_State
|   |   |   \---UML_Game_State.png
|   |   +---piece120.png
|   |   +---piece.png
|   |   +---src
|   |   |   +---UML_PNGs-20170813T183501Z-001.zip
|   |   |   \---UML_XMLs-20170813T183420Z-001.zip
|   |   +---UML_Client.png
|   |   +---UML_Game_Manager.png
|   |   +---UML_Game_State.png
|   |   +---UML_Network.png
|   |   \---UML_Server.png
|   +---release.pdf
|   +---release.tex
|   +---requirements.pdf
|   +---requirements.tex
|   +---testcases.pdf
|   \---testcases.tex
+---Makefile
+---README.md
+---spec
|   +---Evaluation\ form(1).xlsx
|   +---iso-iec-ieee-29148-2011.pdf
|   +---Project\ Description.docx
|   +---Project\ Overview.pptx
|   +---Sample\ Design\ Document.pdf
|   +---Sample\ Requirements\ Document.pdf
|   \---Sample\ Test\ Case\ Document.pdf
+---src
|   +---checkers
|   +---fonts
|   |   \---arcade.h
|   +---fonts.c
|   +---fonts.h
|   +---game
|   |   +---board.c
|   |   +---board.h
|   |   +---display.c
|   |   +---display.h
|   |   +---game_logic.c
|   |   +---game_logic.h
|   |   +---game_manager.c
|   |   +---game_manager.h
|   |   +---move.c
|   |   +---move.h
|   |   +---user_interface.c
|   |   \---user_interface.h
|   +---main.c
|   +---Makefile
|   +---Makefile.am
|   +---network
|   |   +---client.c
|   |   +---client.h
|   |   +---Makefile
|   |   +---server.c
|   |   \---server.h
|   \---test
|       +---Makefile
|       +---TEST_display.c
|       \---TEST_user_interface.c
\---tests
    +---main.c
    +---Makefile.am
    \---tests.c

12 directories, 70 files
\end{lstlisting}

%---------------------------------------------------------------
% Compiling.

Compiling your software is a simple, straightforward process but
it requires that you have \emph{Docker} installed. Docker is
thoroughly described in the accompanying \texttt{design.pdf}
document.

Assuming \emph{Docker} is installed, the build process is quite
straightforward:

\lstset{frame=none}
\begin{lstlisting}
    make game
\end{lstlisting}

Additional ``test'' utilities (demonstrations that individual
components work properly) can be compiled using the following,
but unsupported, commands:

\begin{lstlisting}
    make display
    make logic
    make network
    make user_interface
\end{lstlisting}

%---------------------------------------------------------------
% Installing.

\section{Installing}

The executable(s) produced by the \textbf{Compiling} section are
linked statically, which means that they include all necessary
libraries and objects for the proper execution of our software.

As such, they are ``installed'' to the \texttt{bin/} directory
at the project root, and can be safely copied to any other
\texttt{x86\_64} Linux-based platform.

In the event that our software needs to be run on Windows, two
options exist:

\begin{enumerate}
    \item \textbf{via Docker} -- simply run
          \texttt{make run-con} to execute the binary inside a
          Docker container (works on any platform)
    \item \textbf{recompiling} -- use your favorite C compiler
          to build the requisite libraries (download links are
          provided in the \texttt{Dockerfile}) and then build
          the project source code. The code is compatible with
          Windows.
\end{enumerate}

%---------------------------------------------------------------
% Release Notes.

\section{Release Notes}

There is some odd behavior on some Linux-based systems where the
console display output is misaligned, miscolored, or otherwise
incorrect.

This issue is caused by the improper configuration of the
\texttt{TERM} and/or \texttt{TERMINFO} environment variable(s).
On many popular distributions including Ubuntu and Mint, these
are often incorrectly set.

The simplest remedy is to run the following:

\begin{lstlisting}
    TERM=xterm-color ./bin/game
\end{lstlisting}

%---------------------------------------------------------------
% Unit Testing.

\section{Unit Testing}

We employ the \emph{libcheck} library to perform unit tests. It
is a fairly straightforward framework, taking tests in the
following format:

\lstset{frame=tb}
\begin{lstlisting}
    #include <check.h>
    
    START_TEST(my_test)
    {
	    ck_assert_int_eq(B, 0);
	    ck_assert_int_eq(E, 1);
	    ck_assert_int_eq(F, 2);
    }
    END_TEST
\end{lstlisting}

More information can be found in either the accompanying
\texttt{requirements.pdf} or \texttt{design.pdf} documents.

\textbf{Note}: the \emph{libcheck} library operates differently
from most other ``code coverage'' tools in that, due to limits
with the C programming language, cannot provide a line-by-line
coverity report. As such, the output to our unit tests is the
program return code of \texttt{0}.

%---------------------------------------------------------------
% Code Coverage.

\section{Code Coverage}

As far as code coverage is concerned, we make use of the GNU
coverage too \emph{gcov}. It works like this.

First, source code is compiled with the additional flags:

\lstset{frame=none}
\begin{lstlisting}
    -fprofile-arcs \
    -ftest-coverage \
    -fprofile-dir=./data \
    -fprofile-generate=./data
\end{lstlisting}

Then, there are two choices:

\begin{enumerate}
    \item Run the main game executable, to see if there is any
          ``dead code'' (making sure to test all possible
          functionalities).
    \item Run the unit-testing framework(s), to exercise as much
          of the underlying code as possible, in an effort to
          determine \emph{how much} of the source code is
          \emph{covered} by the unit tests. The accuracy of this
          number strongly depends on the previous point.
\end{enumerate}

Unfortunately, since the client's requirements in the domain of
unit testing and code coverage were no specified, we make no
assumptions as to a hypothetical \emph{minimum} percentage of
code covered. We simply demonstrate that the functionality and
ability to use these tools and run these tests, is properly
implemented. You can verify that this is the case by running the
following command(s):

\lstset{frame=none}
\begin{lstlisting}
    make cover
\end{lstlisting}

This action will perform the aforementioned steps, generating
profiling ``notes'' containing structural information about the
software, as well as profiling statistics when the executable is
tested. The tool \emph{Valgrind} may also be used to profile
C code in much the same fashion. In summary, the concept is
simple: determine code coverage on the unit testing framework.

Of course, unit testing and coverity analysis are definitely
good metrics, but they describe only observable behaviors of the
system and cannot find logic errors better than the humans that
programmed them.

%---------------------------------------------------------------
% Static Analysis.

\section{Static Code Analysis}

We use the tool \emph{cppcheck} to perform static analysis of
our code, as well as the standard C library \emph{MUSL} and any
other header files that we include. It operates by checking each
source file:

\begin{lstlisting}
Checking src/fonts.c ...
[/usr/include/stdio.h:102]: (information) Skipping configuration
    'GCC_PRINTF;printf' since the value of 'printf' is unknown.
    Use -D if you want to check it. You can use -U to skip it explicitly.
[/usr/include/stdio.h:112]: (information) Skipping configuration
    'GCC_SCANF;scanf' since the value of 'scanf' is unknown.
    Use -D if you want to check it. You can use -U to skip it explicitly.
[/usr/include/curses.h:1663]: (information) Skipping configuration
    '__MINGW32__;trace' since the value of 'trace' is unknown.
    Use -D if you want to check it. You can use -U to skip it explicitly.
1/12 files checked 7\% done
Checking src/game/board.c ...
2/12 files checked 19\% done
Checking src/game/display.c ...
[/usr/include/ncurses.h:1663]: (information) Skipping configuration
    '__MINGW32__;trace' since the value of 'trace' is unknown.
    Use -D if you want to check it. You can use -U to skip it explicitly.
3/12 files checked 30\% done
Checking src/game/game_logic.c ...
4/12 files checked 42\% done
Checking src/game/game_manager.c ...
5/12 files checked 53\% done
Checking src/game/move.c ...
6/12 files checked 65\% done
Checking src/game/user_interface.c ...
7/12 files checked 76\% done
Checking src/main.c ...
8/12 files checked 84\% done
Checking src/network/client.c ...
[./src/network/client.c:88]: (information) Skipping configuration
    'GCC_PRINTF;printf' since the value of 'printf' is unknown.
    Use -D if you want to check it. You can use -U to skip it explicitly.
[./src/network/client.c:92]: (information) Skipping configuration
    'GCC_PRINTF;printf' since the value of 'printf' is unknown.
    Use -D if you want to check it. You can use -U to skip it explicitly.
[./src/network/client.c:136]: (information) Skipping configuration
    'GCC_PRINTF;printf' since the value of 'printf' is unknown.
    Use -D if you want to check it. You can use -U to skip it explicitly.
[./src/network/client.c:146]: (information) Skipping configuration
    'GCC_PRINTF;printf' since the value of 'printf' is unknown.
    Use -D if you want to check it. You can use -U to skip it explicitly.
[./src/network/client.c:150]: (information) Skipping configuration
    'GCC_PRINTF;printf' since the value of 'printf' is unknown.
    Use -D if you want to check it. You can use -U to skip it explicitly.
9/12 files checked 88\% done
Checking src/network/server.c ...
[./src/network/server.c:117]: (information) Skipping configuration
    'GCC_PRINTF;printf' since the value of 'printf' is unknown.
    Use -D if you want to check it. You can use -U to skip it explicitly.
[./src/network/server.c:146]: (information) Skipping configuration
    'GCC_PRINTF;printf' since the value of 'printf' is unknown.
    Use -D if you want to check it. You can use -U to skip it explicitly.
[./src/network/server.c:152]: (information) Skipping configuration
    'GCC_PRINTF;printf' since the value of 'printf' is unknown.
    Use -D if you want to check it. You can use -U to skip it explicitly.
[./src/network/server.c:176]: (information) Skipping configuration
    'GCC_PRINTF;printf' since the value of 'printf' is unknown.
    Use -D if you want to check it. You can use -U to skip it explicitly.
[./src/network/server.c:188]: (information) Skipping configuration
    'GCC_PRINTF;printf' since the value of 'printf' is unknown.
    Use -D if you want to check it. You can use -U to skip it explicitly.
[./src/network/server.c:193]: (information) Skipping configuration
    'GCC_PRINTF;printf' since the value of 'printf' is unknown.
    Use -D if you want to check it. You can use -U to skip it explicitly.
[./src/network/server.c:197]: (information) Skipping configuration
    'GCC_PRINTF;printf' since the value of 'printf' is unknown.
    Use -D if you want to check it. You can use -U to skip it explicitly.
[./src/network/server.c:206]: (information) Skipping configuration
    'GCC_PRINTF;printf' since the value of 'printf' is unknown.
    Use -D if you want to check it. You can use -U to skip it explicitly.
10/12 files checked 92\% done
Checking src/test/TEST_display.c ...
11/12 files checked 96\% done
Checking src/test/TEST_user_interface.c ...
12/12 files checked 100\% done
\end{lstlisting}

Static code analysis may reveal bits of \emph{truly} dead code,
or code that is entirely inaccessible during normal execution of
the program. It can reveal logic errors, as well, but there is a
tradeoff between performance and accuracy. Not even the slowest,
most thorough, expensive tools can detect all errors with 100\%
certainty, but a static code analyzer is definitely a step in
the right direction.

%---------------------------------------------------------------
% Document end.

\end{document}